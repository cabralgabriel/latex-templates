% CONFIGURAÇÕES DE APARÊNCIA DO PDF FINAL---------------------------------------
\makeatletter
\hypersetup{%
    portuguese,
    colorlinks=false,   % true: "links" coloridos; false: "links" em caixas de texto
    linkcolor=blue,    % Define cor dos "links" internos
    citecolor=blue,    % Define cor dos "links" para as referências bibliográficas
    filecolor=blue,    % Define cor dos "links" para arquivos
    urlcolor=blue,     % Define a cor dos "hiperlinks"
    breaklinks=true,
    pdftitle={\@title},
    pdfauthor={\@author},
    pdfkeywords={abnt, latex, abntex, abntex2}
}
\makeatother

% REDEFINIÇÃO DE LABELS---------------------------------------------------------
\renewcommand{\algorithmautorefname}{Algoritmo}
\def\equationautorefname~#1\null{Equa\c c\~ao~(#1)\null}

\crefname{table}{Tab.}{Tabs.}
\Crefname{table}{Tabela}{Tabelas}

% CRIA ÍNDICE REMISSIVO---------------------------------------------------------
\makeindex

% HIFENIZAÇÃO DE PALAVRAS QUE NÃO ESTÃO NO DICIONÁRIO---------------------------
\hyphenation{%
    qua-dros-cha-ve
    Kat-sa-gge-los
}

% CENTRALIZAR CONTEÚDO DA CÉLULA USANDO M
\newcolumntype{M}[1]{>{\centering\arraybackslash}m{#1}}

% CONFIGURA TAMANHO LEGENDA DAS IMAGENS
\captionsetup{font=small, singlelinecheck=true, skip = -1pt}

%
\allowdisplaybreaks
