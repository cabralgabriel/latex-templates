% ----------------------------------------------------------------------------
%                    CONFIGURAÇÕES DE APARÊNCIA DO PDF FINAL
% ----------------------------------------------------------------------------
\makeatletter
\hypersetup{%
    portuguese,
    colorlinks=false,  % Define se os "links" ficarão coloridos
    linkcolor=blue,    % Define cor dos "links" internos
    citecolor=blue,    % Define cor dos "links" para as referências bibliográficas
    filecolor=blue,    % Define cor dos "links" para arquivos
    urlcolor=blue,     % Define a cor dos "hiperlinks"
    breaklinks=true,   % Ativa quebra de links longos em várias linhas
    pdftitle={\@title},
    pdfauthor={\@author},
    pdfkeywords={abnt, latex, abntex, abntex2}
}
\makeatother
\makeindex % Cria indice remissivo

% ----------------------------------------------------------------------------
%                           OUTRAS CONFIGURACOES
% ----------------------------------------------------------------------------

% Renomeação de Labels para Referências Cruzadas
\renewcommand{\algorithmautorefname}{Algoritmo}
\def\equationautorefname~#1\null{Equa\c c\~ao~(#1)\null}

\crefname{chapter}{Cap.}{Caps.}
\Crefname{chapter}{Capítulo}{Capítulos}

\crefname{section}{Sec.}{Sec.}
\Crefname{section}{Seção}{Seções}

\crefname{subsection}{Subsec.}{Subsec.}
\Crefname{subsection}{Subseção}{Subseções}

\crefname{table}{Tab.}{Tabs.}
\Crefname{table}{Tabela}{Tabelas}

% Hifenização de palavras que não estão no dicionário
\hyphenation{%
    qua-dros-cha-ve
    Kat-sa-gge-los
}

\newcolumntype{M}[1]{>{\centering\arraybackslash}m{#1}} % Centralizar conteúdo da célula
\captionsetup{font=small, singlelinecheck=true, skip = -1pt} % Configura a legenda dos elementos dos /dados
\allowdisplaybreaks % Permite quebra de página no ambiente align
