% ----------------------------------------------------------------------------
%                                   REFERÊNCIAS
% ----------------------------------------------------------------------------
\usepackage[%
    alf,
    abnt-emphasize=bf,
    bibjustif,
    recuo=0cm,
    abnt-url-package=url,       % Utiliza o pacote url
    abnt-refinfo=yes,           % Utiliza o estilo bibliográfico abnt-refinfo
    abnt-etal-cite=1,
    abnt-etal-list=0,
    abnt-etal-text=it,           % Deixa expressao et al em italico
    abnt-thesis-year=final
]{abntex2cite}                  % Configura as citações bibliográficas conforme a norma ABNT

% ----------------------------------------------------------------------------
%                                    PACOTES
% ----------------------------------------------------------------------------

% Codificação e Fonte
\usepackage[utf8]{inputenc}             % Codificação do documento
\usepackage[T1]{fontenc}                % Seleção de código de fonte
\usepackage{times}                      % Usa a fonte Times
%\usepackage[scaled]{helvet}            % Usa a fonte Helvetica
%\usepackage{palatino}                  % Usa a fonte Palatino
%\usepackage{lmodern}                   % Usa a fonte Latin Modern
\usepackage{ae, aecompl}                % Fontes de alta qualidade
\usepackage{latexsym}                   % Símbolos matemáticos
\usepackage{icomma}                     % Uso de vírgulas em expressões matemáticas
\usepackage{microtype}                  % Melhora a justificação do documento

% Tabelas e Imagens
\usepackage{booktabs}                   % Réguas horizontais em tabelas
\usepackage{color, colortbl}            % Controle das cores
\usepackage{float}                      % Necessário para tabelas/figuras em ambiente multi-colunas
\usepackage{graphicx, caption}          % Inclusão de gráficos e legenda
\usepackage{lscape}                     % Permite páginas em modo "paisagem"

% Matemática
\usepackage[tbtags]{amsmath}
\usepackage{amsfonts, amssymb}
\usepackage{subeqnarray}                % Permite subnumeração de equações
\usepackage[algoruled, portuguese]{algorithm2e}  % Permite escrever algoritmos em português
\usepackage{regexpatch}                 % Patching comandos LaTeX

% Estilo e Formatação
\usepackage{indentfirst}                % Indenta o primeiro parágrafo de cada seção
\usepackage{fancyhdr}                   % Cabeçalhos e rodapés personalizados
\usepackage{titlesec}                   % Controle de títulos de seção
\usepackage{varwidth}                   % Ajuste do tamanho da caixa

% Outros
\usepackage{lastpage}                   % Para encontrar última página do documento
\usepackage{verbatim}                   % Apresentar texto como escrito no documento
\usepackage{lipsum}                     % Geração de texto de preenchimento
\usepackage{tocloft}                    % Personalizar formatação de listas
\usepackage{etoolbox}                   % Ferramentas para manipulação de comandos
\usepackage[nameinlink,capitalize]{cleveref} % Referências cruzadas inteligentes
\usepackage{pdfpages}                   % Inclui documentos PDF externos